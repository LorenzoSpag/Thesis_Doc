\chapter{Conclusion}

In this thesis various methods were used in an attempt to predict the prognosis of \covid patients. 

Regularized regression was used to predict clinical outcome using various families of variables to compare the information hidden in each of them and to evaluate if CT exams add any value to a small set of clinical variables.

Random forest classifiers were used with the same aim. As a by-product these two methods can be compared to see which, given the same data, can extract the most information.

Survival analysis was used, mostly by itself, to see if the data could be divided in smaller groups with different survival functions.

In the first two lines of development it was found that, in the specific case of data available for this thesis, models built on radiomic features perform in a statistically equivalent way to models built from clinical variables.
%This was taken as a (grim) reminder that the quality of the segmentations is very relevant in determining the results obtained by radiomics analysis and that current commercial segmentation techniques based on thresholding and region growing are not yet ready to face the problem posed by \covid pneumonia lungs.

%However the results obtained can be looked at from another perspective.
%The method developed, which also shows potential for improvement, has shown that models using radiomic features have performances on par with those obtained using clinical variables.
This means that, especially in times of system overload due to increased accesses during a pandemic, this system could be integrated in PACS systems of the hospital to bring to attention some of the patients in worst condition.
This, especially thanks to the objectivity of radiomics, the user-independence and, most of all, to the semi-automatic nature of the pipeline could be done on a large scale to aid in hospitals that don't have the facilities , or that lack the personnel, to analyze in detail all cases.
It has also been hypothesised that a more careful handling of images during the segmentation phase, perhaps obtainable with instruments specifically developed for \covid, could lead to improvements in performances of the model.

Regarding survival analysis it was found that the cohort can be divided in parts using the hazard predicted by a Cox Proportional Hazard model, and it was also found that a time-division driven by the "wave" definition of the pandemic produces significantly different survival curves.

All of that said, some directions in which future works could start from this thesis are:

\begin{itemize}
\item The implementation of a \covid specific segmentation method using the vast amount of available data, similar to what has been done in \cite{Biondi}.
\item It would be interesting to test the pipeline, as well as the deriving models, on manually segmented CT scans to compare the results and performances.
\item Given the statistical difference found in the survival of patients in the first and second waves, defined here as before and after 20/07/2020, it might be very interesting to investigate the causes of this finding and to prospectively continue this analysis with the data from the third wave and eventual next ones that might occur.
\item Using CT scans acquired at different points in the course of the illness it would be very interesting to implement a variation of the pipeline along the direction of delta-radiomics.
\end{itemize}
