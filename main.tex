\documentclass[12pt,a4paper]{report}

\usepackage[utf8]{inputenc} %usa il formato utf8 per l'encoding
\usepackage[T1]{fontenc} % usa un font ragionevole
\usepackage{lmodern} % migliora alcuni dettagli della visualizzazione del font
\usepackage{todonotes} %questo serve per mettere le note

\usepackage[nottoc]{tocbibind}%mette la bibliografia nella table of contents
\usepackage[graphicx]{realboxes}%per ruotare le tabelle
\usepackage{adjustbox}
\usepackage{geometry}
\usepackage{longtable}
\usepackage{float}
\usepackage{titlesec}
\usepackage{amssymb,amsmath}
\usepackage{amsthm}
\usepackage[]{graphicx} \graphicspath{ {Fig/} }
\usepackage{caption}
\usepackage[]{xcolor}
\usepackage{booktabs, multirow}
\usepackage[colorlinks,linkcolor=black,urlcolor=blue,citecolor=gray]{hyperref}
\usepackage{natbib}
\usepackage{url}
\usepackage{lineno}
\usepackage{subfig}
\usepackage{graphicx}

\usepackage{subfig}
\usepackage[italian, english]{babel}  %%% !!! Cancellare 'english' per avere "Sommario, Capitolo..." !!!

%------------------------------------------------------------------------------------
% questi non dovrebbero servirti, in particolare con i pacchetti che ho aggiunto
\newcommand{\eg}{\`e}
\newcommand{\ea}{\'e}
\newcommand{\Eg}{\`E}
\newcommand{\og}{\`o}
\newcommand{\ag}{\`a}
\newcommand{\ug}{\`u}
\newcommand{\ig}{\`{\i}}
\newcommand{\R}{\textsf{R}}
\newcommand{\Fortran}{\textsf{Fortran}}
%------------------------------------------------------------------------------------
\definecolor{MyGray}{rgb}{0.30,0.31,0.32} % definizione grigio
\definecolor{MyDarkBlue}{rgb}{0.,0.08,0.5} % blu scuro
\definecolor{MyLightBlue}{rgb}{0.2,0.2,1.0} % blu chiaro
\definecolor{MyDarkRed}{rgb}{0.5,0.04,0} % rosso scuro
\definecolor{MyDarkGreen}{rgb}{0.0,0.4,0.08} % verde scuro
\definecolor{SectionBlue}{rgb}{0.16,0.32,0.75} % blu per sezioni
\definecolor{SubSectionBlue}{rgb}{0.0,0.53,0.74} % blu per sottosezioni
\definecolor{SubSubSectionBlue}{rgb}{0.0,0.58,0.65} % blu per sottosottosezioni
%------------------------------------------------------------------------------------
\titleformat{\chapter}[display]
  {\color{SectionBlue}\normalfont\LARGE\bfseries}{\chaptertitlename\ \thechapter}{20pt}{\Huge}
\titleformat{\section}
{\color{SectionBlue}\normalfont\Large\bfseries}
{\color{SectionBlue}\thesection}{1em}{}
\titleformat{\subsection}
{\color{SubSectionBlue}\normalfont\Large\bfseries}
{\color{SubSectionBlue}\thesubsection}{1em}{}
\titleformat{\subsubsection}
{\color{SubSubSectionBlue}\normalfont\large\bfseries}
{\color{SubSubSectionBlue}\thesubsubsection}{1em}{}
\DeclareCaptionFont{myblue}{\color{SubSectionBlue}}
\captionsetup{labelfont={myblue,bf}}
%------------------------------------------------------------------------------------
\newcommand{\myboldverb}[1]{{\color{MyDarkBlue}\bfseries{#1}}} % per enfasi in grassetto
\newcommand{\myverb}[1]{{\color{MyDarkBlue}\texttt{#1}}} % in evidenza
\newcommand{\myurl}[1]{{\color{MyDarkGreen}\url{#1}}} % per url
\newcommand{\quotesection}[1]{{\color{SectionBlue}\textbf{Sezione \ref{#1}}}}
\newcommand{\quotesubsection}[1]{{\color{SubSectionBlue}\textbf{Sezione \ref{#1}}}}
\newcommand{\quotesubsubsection}[1]{{\color{SubSubSectionBlue}\textbf{Sezione \ref{#1}}}}
\newcommand{\quotefigure}[1]{{\color{MyDarkGreen}\textbf{Figura \ref{#1}}}}
\newcommand{\quotetable}[1]{{\color{MyDarkGreen}\textbf{Tabella \ref{#1}}}}

\newcommand{\covid}{\textit{Sars-COVID19}}
%--------------------------------------------------------------------------------------
\theoremstyle{definition}
\newtheorem{definition}{Definition}[section]

\theoremstyle{remark}
\newtheorem*{remark}{Remark}

\graphicspath{ {./imgs/} }
%\setlength\parskip{1\baselineskip} % meglio separare con l'indentazione, non con lo spazio interlinea
%------------------------------------------------------------------------------------
\geometry{
	paper=a4paper, % Change to letterpaper for US letter
	inner=2.5cm, % Inner margin
	outer=3cm, % Outer margin
	bindingoffset=0.5cm, % Binding offset
	top=2.5cm, % Top margin
	bottom=3cm, % Bottom margin
	%showframe, % Uncomment to show how the type block is set on the page
}


%------------------------------------------------------------------------------------
%------------------------------ HERE WE GO ------------------------------------------
%------------------------------------------------------------------------------------


\begin{document}
\title{THESIS TITLE}
\author{Lorenzo Spagnoli}
\date{19th May 2021}

%----------------------------- Frontispiece
\newgeometry{margin=0.8in}
\begin{titlepage}
    \begin{center}
        \vspace*{0.2cm}
        {\fontsize{19pt}{20pt}\selectfont \textbf{Alma Mater Studiorum $\cdot$ University of Bologna}\par}

        \par\noindent\rule{\textwidth}{.5pt}
        \rule[.5\baselineskip]{\textwidth}{1.5pt}
        \vspace{0.8cm}

        \normalfont

        Department of Physics and Astronomy

        Master Degree in Physics

        \Large
        \vspace{5cm}
        {\fontsize{22.5pt}{22.5}\textcolor{MyDarkRed}{\textbf{THESIS TITLE}}}


        \vspace{4.5cm}
        \begin{minipage}[t]{0.64\textwidth}
        \begin{flushleft}
        {\fontsize{16pt}{16pt}\textbf{Supervisor:} \\ \textbf{\textcolor{MyDarkRed}{Prof. Enrico Giampieri}}} \\
        {\fontsize{16pt}{16pt}\textbf{Co-Supervisor:} \\ \textbf{\textcolor{MyDarkRed}{Dott.sa Lidia Strigari}}}
        \end{flushleft}
        \end{minipage}
        \begin{minipage}[t]{0.34\textwidth}
        \begin{flushright} \Large
        \textbf{Submitted by:} \\
        \textbf{\textcolor{MyDarkRed}{Lorenzo Spagnoli}} \\
        \end{flushright}
        \end{minipage}\\

        \vfill
        \vspace{0.3cm}
        \small

        Academic Year {\textcolor{MyDarkRed} {2020-2021}}
    \end{center}
\end{titlepage}
\restoregeometry
\vspace*{5cm}
\begin{flushright}
  {\parbox{4.2cm}{\textit{Dedication...}}}
\end{flushright}

\thispagestyle{empty}
%\newpage
\listoftodos
%-------------------------------------ABSTRACT
%\thispagestyle{empty}
\vspace*{5cm}
\renewcommand{\abstractname}{\color{MyDarkBlue}{Abstract}}
\begin{abstract}
%Useful diagnostic vs prognostic and data- vs law-driven given in \cite{1} page 18.


\todo[inline]{bisogna dividere i pagrafi, e consiglio una sola frase per riga. Dividere i paragrafi aiuta anche con il layout di tabelle e figure}.
Since the start of 2020 \covid has given rise to a world-wide pandemic
In an attempt to slow down the fast and uncontrollable spreading of this disease various prevention and diagnostic methods have been developed.
In this thesis, out of all these various methods, the attention is going to be put on Machine Learning methods used to predict prognosis that are based, for the most part, on data originating from medical images.

The techniques belonging to the field of radiomcs will be used to extract information from images segmented using a software available in the hospital that provided the clinical data as well as the images.
The usefulness of different families of variables will be evaluated through their performance in the methods used, namely Lasso regularized regression and Random Forest.
Dimensionality reduction techniques will be used to attain a better understanding of the dataset at hand.

Following a first introductiory chapter in the second a basic theoretical overview of the necessary core concepts that will be needed throughout this whole work will be provided and then the focus will be shifted on the various methods and instruments used in the development of this thesis.
The third is going to be a report of the results and finally some conclusions will be derived from the previously presented results.
It will be concluded that the segmentation and feature extraction step is of pivotal importance in driving the performance of the predictions.
In fact, in this thesis, it seems that the information from the images adds no significant information to that derived from the clinical data.
This can be taken as a symptom that the more complex \covid cases are still too difficult to be segmented automatically, or semi-automatically by untrained personnel, which will lead to counter-intuitive results further down the analysis pipeline.

\end{abstract}
%-------------------------------------INDEX
\setcounter{page}{1}
\pagenumbering{roman}
\tableofcontents{}
%\vspace*{2cm}
%\setcounter{page}{1}
%\pagenumbering{arabic}

\linenumbers
%%%%%%%%%%%%%%%%%%%%%%%%%%%%%%%%%%%%% Chapter 1: introduction
\chapter{Introduction}
\vspace*{0.5cm}
\setcounter{page}{1}
\pagenumbering{arabic}

%Since the start of 2020 \textit{Sars-COVID19} has initiated a world-wide pandemic. In an attempt to slow down the fast and uncontrollable spread of this disease various prevention and diagnosis methods have been developed. In this work, out of all these various methods, the attention is going to be put on the possible developement diagnostic methods related to medical images, be they automatic or semi-automatic, to be intended either as Clinical Decision Support Systems (CDSS) or as a quick evaluation to completely avoid human analysis.
%In this light this first introductory chapter is going to start from a basic theoretical overview of the necessary core concepts that will be needed throughout this whole work such as definition of an image and imaging methods, with particular attention to those used in the medical field. This will be followed by an introduction to Artificial Intelligence (AI) and some Machine Learning techniques. Finally some of the concepts from the two previous topics will be treated jointly under the discipline of radiomics, which will be defined and explored as necessary.

Nowadays everybody knows of  \textit{Sars-COVID19} which, since the start of 2020, has made necessary a few world-wide quarantines forcing everybody in self-isolation. It is also well known that, among the main complications and features of this virus, symptoms gravity as well as the rate of deterioration of the conditions are some of the most relevant and problematic. In some cases asymptomatic or near to asymptomatic people may, in the span of a week, get to conditions that require hospital admission. This peculiarity is also what heavily complicates the triage process, since trying to predict with some degree of accuracy the prognosis of the patient at admission is a thoroughly complex task. In this thesis the aim will be to use data, specifically including data that cannot be easily interpreted by humans, to try various methods to predict a couple of clinical outcomes, namely the death of the patient or the admission in the Intensive Care Unit (ICU), while assessing their performance. These analyses will be carried out on a dataset of 434 patients with different variables associated to every person. A part of the variables, which will be called clinical and radiological, are defined by humans and are generally discrete in nature but mostly boolean. The most part of the available variables, however, will be image-derived following the approaches used in the field of radiomics. While the utility of clinical variables, such as age, obesity and history of smoking, is very straightforward it's interesting and helpful to understand the basis behind the utility of radiomic and radiological features.
Generally speaking it's clear that images have the ability to convey a slew of useful images, this is expecially true in the medical field where digital images are used to inspect also the internal state of the patient giving far more detailed information than that obtainable by visual inspection at the hand of medical professionals. Among the ways in which \textit{Sars-COVID19} can manifest himself the one that is most relevant to the scopes of this thesis pneumonia and the complications that stem from it. Some of these complications, which are not specific of \textit{Sars-COVID19} but can happen in any pneumonia case, display very peculiar patterns when visualizing the lungs through CT exams.
These patterns are due to the pulmonary response to inflammation which may lead to thickening of the bronchial and alveolar structures up to pleural effusions and collapsed lungs. Without going too much in clinical detail what is of interest is how these condition manifest themselves in the CT exams:

\begin{enumerate}
        \item \textbf{Ground Glass Opacity(GGO)}:\newline Small diffused changes in density of the lung structure cause a hazy look in the affected region. This complicates the individuation of pulmonary vessels.
			\begin{figure}[htbp]
				\centering
				\includegraphics[width=0.66\linewidth]{GGO.jpg}
				\caption{Example of GGO\label{GGOImage}}
			\end{figure}
        \item \textbf{Lung Consolidations}:\newline Heavier damage reflects in whiter spots in the lung as the surface more closely resembles outside tissue instead of normal air. The consolidation refer to presence of fluid, cells or tissue in the alveolar spaces	
        \item \textbf{Crazy paving}:\newline When GGOs are superimposed with inter-lobular and intra-lobular septal thickening.
			\begin{figure}[htbp]
				\centering
				\subfloat[][a]{\includegraphics[width=0.35\linewidth]{CollapsedLung.jpg}}
				\subfloat[][a]{\includegraphics[width=0.33\linewidth]{PleuralEffusion.png}}
				\caption{Differences between a collapsed lung (a) and pleural effusion(b)}
			\end{figure}
	\item \textbf{Collapsed Lungs and Pleural Effusion}:\newline Both of these manifest themself as regions of the lungs that take the same coloring as that of tissue outside the lung. The main difference between the two is that collapsed lungs are somewhat rigid structures, they can occur in singular lobes of the lung and stay where they occur. Pleural effusions, however, are actually fluid being located in the lung instead of air. As such these lesions usually are located 'at the bottom' of the lung in which they happen and migrate to the lowest part of the lung according to the position of the patient.
\end{enumerate}

Having these manifestation it's clear that they are mainly textural and intensity-like changes in the normal appearance of the lungs. However, whereas these properties can be easily described in a qualitative and subjective way, it's rather complex to describe them in a quantitative and objective way. The field of radiomics, when coupled with digital images and preprocessing steps which must include image segmentation, is exactly what undertakes this daunting task. Radiomics comes from the the combination of radiology and the suffix \textit{-omics}, which is characteristic of high-throughput methods that aim to generate a large number of numbers, called biomarkers or features, as such it uses very precise and strict mathematical definitons to quantify in various ways either shape, textural or intensity based properties of the radiological image under analysis.
Given the large numerosity of the features produced by radiomics it's necessary to analyze these kinds of data with methods that rely on Machine Learning and their ability to address high-dimensional problems, be it in a supervised or unsupervised way, in a rather fast and accurate way.
Starting from these premises this thesis will be divided in a few chapters and sections. The first step will be taken by providing the general theoretical background regarding the aforementioned topics and techniques, this will be followed by a desscription of the data in use as well as a presentation of the analysis methods and resources used. Finally the results of the methods described will be presented and from them a set of concluding remarks will be set forth.


%%%%%%%%%%%%%%%%%%%%%%%%%%%%%%%%%%%%% Chapter 2:
\chapter{Materials and methodologies}
In this section there's going to be an explanation of the dataset as well as instruments and methodologies used to analyze it's properties, as such the first step is going to be an in depth discussion of the data available and a general overview of the final use. The following step is going to be a description of the preliminary work done to the data itself and to the results of this preliminary analysis in order to select the important features. The final step of this chapter is going to be an explanation of the methods used to derive the final results and to evaluate them.

\section{Data and objective}
\label{chap:freefree}
The objective of this thesis is to find how different models for survival or clinical outcome behave as the input feature change. In other words a first part is the development of one or more models for survival analysis followed by a sensitivity analysis study of the developed models. The main focus is going to be on clinical images, i.e. lung CTs, used in conjunction with clinical and laboratory labels which are obtainable as soon as the patient is admitted in the hospital facilities. All clinical images are going to be analysed through the lens of \textbf{radiomics} while all clinical labels are provided as is by clinical professionals.
All images are non segmented, as such all of them are going to be semi-automatically segmented via a new software being tested in the medical physics department called \textit{\textbf{Sophia}}, which will be briefly explained shortly. Statistical analysis are going to be performed mainly in python using libraries such as scikit-learn, pandas, numpy, scipy while the graphical part of the analysis is done with either seaborn or matplotlib. \large{For the sake of time and simplicity some analysis was carried out in R language using the following libraries..... NON ANCORA MA SE SUCCEDE E' PRONTO DA AGGIUNGERE}
The starting dataset was a list of all the patients that, from 02/2020 to 05/2021, were hospitalized as COVID-19 positive inside the facilities of \textit{Azienda ospedaliero-universitaria di Bologna - Policlinico Sant'Orsola-Malpighi}. As far as exclusion criteria go the main exclusion criteria, except unavailability of the feature related to the patient, was visibly damaged or lower quality images, for example images with cropped lungs. The first set of selection criteria were:

\begin{itemize}
\item All patients that had undergone a CT exam which was retrievable via the PACS system of \textit{Azienda ospedaliero-universitaria di Bologna - Policlinico Sant'Orsola-Malpighi}
\item All patients that had a all of the clinical and laboratory features, listed in fig:\ref{fig:ClinicalFeatures}, suggested by Lidia Strigari \footnote{She is, as of the writing of this thesis, the head of the Medical Physics department in the S. Orsola hospital. These suggestions were given to her by clinical professionals.}
\item Since all patient had at least 2 CT exams only the closest date to the hospital admission date was taken. When more exams were performed on the same date all of them were initially taken. At first only chest or abdomen CTs were taken regardless of the acquisition protocol used.
\end{itemize}

\begin{figure}[H]
  		\includegraphics[width=1\textwidth]{Clinical_Feature_Description.png}
        \caption{Description of clinical label dataset, in sets of four columns there's the clinical feature, the total count of occurences, the percentage over the final dataset and the possible values the feature could take \label{fig:ClinicalFeatures}}
\end{figure}

Most clinical features are pretty self-explanatory, for those that were obscure to me as an outsider and not otherwise explained in \ref{Clinical_Covid_Manifestations}, I'm going to provide a very simple explanation:

\begin{enumerate}
\item DNR : Acronym for "Do Not Resuscitate", used to indicate the wish of the patient or their relatives that cardiac massage not be performed in case of cardiac arrest. 
\item NIV: Acronym for "Non Invasive Ventilation", it's a form of respiratory aid provided to patients.
\item cPAP: Acronym for  "continuous Positive Airway Pressure", another form of respiratory aid.
\item ICU: Acronym for "Intensive Care Unit". When patients are in really severe conditions they are treated in these facilities.
\item Clinical Scores: When available values from laboratory analyses and/or patient conditions are summarised in scores that represent the gravity of the state of the patient, as such these can be somewhat correlated and will be treated as comprehensive values to substitute an otherwise large set of obscure clinical features. At admission, or closely thereafter, a set of clinical questions regarding the patient receives a yes or no answer, each answer has an additive contribution towards the final value of the score.These scores differ in how much they add for each condition and the set of symptoms the check for.
	\begin{enumerate}
		\item MulBSTA: This score acconts for \textbf{Mul}tilobe lung involvement, absolute \textbf{L}ymphocyte count, \textbf{B}acterial coinfection, hystory of \textbf{S}moking, history of hyper\textbf{T}ension and \textbf{A}ge over 60 yrs. \cite{MulBSTA}
		\item MEWS: Modified Early Warning Score for clinical deterioration. Computed considering systolic blood pressure, heart rate, respiratory rate, temperature and AVPU(Alert Voice Pain Unresponsive) score. \cite{MEWS}
		\item CURB65: \textbf{C}onfusion, blood \textbf{U}rea Nitrogen or Urea level, \textbf{R}espiratory Rate, \textbf{B}lood pressure, age over \textbf{65} years. This score is specific for pneumonia severity \cite{CURB65}
		\item SOFA:  \textbf{S}equential  \textbf{O}rgan  \textbf{F}ailure  \textbf{A}ssessment score. Considers various quantities from all systems to assess the overall state of the patient, PaO$\_2$/FiO$\_2$\footnote{Very unrefined yet widely used indicator for lung disfunction} for respiratory system, Glasgow Coma scale\footnote{GCS for short, proposed in 1974 by Graham Teasdale and Bryan Jennet. Evaluates what kind of stymulus is necessary to obtain motor and verbal reactions in the patient as well as what's necessary for the patient to open their eyes} for nervous, mean pressure for cardiovascular, Bilirubin levels for liver, platelets for coagulation and creatine for kidneys \cite{SOFA}
		\item qSOFA:  \textbf{q}uick SOFA. Only considers pressure, high respiratory rate and the low values in the Glasgow scale.
	\end{enumerate}	
\end{enumerate}

This procedure produced a starting cohort of $\sim$700 patients which, having all various images available, created a huge set of $\sim$2200 CT scans. Since this analysis is focused on radiomics there is an evident need for as much consistency as possible in the images analysed. For this reason all CTs taken with medium of contrast were excluded, since they would have brightnesses not justified, and for every patient only images with thin slice reconstruction were considered. More specifically only images with slice thickness of 1 o 1.25 mm\footnote{This meant that only exams called 'Parenchima' or  'HRCT' were included. Throughout the internship 'parenchima' has always appeared in contrast with 'mediastino'. These two keywords are used in the phase of reconstruction of the raw data to identify reconstructions with specific properties. Parenchima is used for finer reconstruction of lung specifically, the requiring professional uses these images to look for small nodules with very high contrast and, to do so, the reconstruction allows some noise to achieve the best resolution possible. Mediastino is used in the lung, as well as other regions, to look for bigger lesions but with low contrast. As such the 'mediastino' reconstruction compromises a worse spatial resolution for a better display of contrast, visually speaking the first images are more coarse and noisy while the second are smoother. It should be noted that even with the same identifier, be it HRCT parenchima or others, the machines on which the exams were made were different and had different proprietary convolutional kernels used for reconstruction.} along the z-axis were taken into consideration, which meant excluding all the 1.5,2,2.5 and 5 mm slice thicknesses.
Since all these images were segmented by me and two other students using a semiautomatic segmentation tool provided by the hospital it sometimes happened that manual corrections were necessary, these were performed only in very obvious and simple cases while, in all remaining cases, the patient was dropped out of the study. 
Overall this left the final study cohort to be composed of 436 patients, all descriptions and analyses are related to this cohort.































%%%%%%%%%%%%%%%%%%%%%%%%%%%%%%%%%%%%% Chapter 3
\chapter{Results}
In this chapter the result obtained with the methods explained in the previous chapters will be briefly presented, to ease in weaving through the quantity of results a structure of the result presentation is reported in fig \ref{ResultFlowchart} and a final summary will be provided in the following chapter.

\begin{figure}[htbp]
  		\includegraphics[width=1\textwidth]{Result_flowchart.png}
        \caption{Logical structure used in the presentation of the results.\label{ResultFlowchart}}
\end{figure}

\section{Predicting and classifying the outcome \death}
First step in reporting the results is going to be using \death as the clinical outcome of interest for either Lasso regularized regression or Random Forest classifier.

\subsection{Feature selection through Lasso regularization and clinical outcome prediction using regression}
When it comes to lasso regression usually graphs are reported that show the convergence of the parameters to the final value. Since the real information of this process is the value to which the coefficients converge only these values will be presented in tables and the ROC curves, with respective AUCs, will be provided. An example of the aforementione graph is the one visible in Figure \ref{LassoParam}.


\begin{figure}[htbp]
  		\includegraphics[width=1\textwidth]{Lasso_rad.png}
        \caption{Example of graph representing the convergence of the coefficients in a lasso procedure. The top graph represents how the values of the weights of the input features change as the Lasso hyperparameter $\alpha$ changes, the top axis label indicates at each point how many features would have weight different from zero. The bottom graph shows the curves that represent the behaviour of the mean squared error of the model on the validation set, the test set and the training set. One of the ways to find the optimal value of the $\alpha$ parameter is to find the value that minimizes the mean squared error relative to the validation set. The vertical dashed line is a graphical representation of how the values for the feature weight is chosen. This particular graph is relative to only radiomic features and comes from the regularization of a model that uses \death as target variable.\label{LassoParam}}
\end{figure}

Finally it seems useful to report Table \ref{tab:cont_tab} the contingency table that gives an idea on how superimposed the clinical outcomes on \death and \icu are.

\begin{table}
\caption{Contingency table that quantifies overlap between individual accessed in the ICU and Dead individuals.\label{tab:cont_tab}}
\centering
\begin{tabular}{l|rr}
\toprule
{} & \multicolumn{2}{l}{\icu} \\
 &       0 &   1 \\
\death&         &     \\
\midrule
0     &     311 &  47 \\
1     &      48 &  30 \\
\bottomrule
\end{tabular}
\end{table}

Starting to predict the death outcome, using Lasso regularized regression, of the patient different groups of features have been used, first of all only the radiomic features have been used. The ROC curve obtained with the radiomic features is the one in Figure \ref{RocDeathRad}. The curve in bold is an average curve obtained by aggregating the ten curves relative to each of the folds used in testing and the gray band represents a $\pm$ 1 standard deviation.

\begin{figure}[htbp]
	\centering
  		\includegraphics[scale=0.25]{ROC_CV_rad.png}
        \caption{ROC curves obtained with crossvalidation procedure using the radiomic features alone. In bold is the mean ROC with gray bands of width equal to the standard deviation. \label{RocDeathRad}}
\end{figure}

The model that was built using the coefficients reported in \ref{tab:ParamRad} reaches a AUC = 0.76$\pm$0.09. The features inside the tabular, as well as those in the tabulars that follow, will have the coefficients in descending order by absolute value so that the top features are the most relevant within it's relative model.

\begin{table}
\caption{Coefficients used in the linear combination estimated by a Lasso regularization relative to the radiomic features in modelling \death. All values are in descending order of absolute value\label{tab:ParamRad}}
\centering 
	\begin{tabular}{lr}
		\toprule
		Feature Name &   Importance \\
		\midrule
		Intercept                           &                      0.178899 \\
		10th intensity percentile           &                     -0.125094 \\
		Intensity-based interquartile range &                      0.103349 \\
		Complexity                          &                     -0.102924 \\
		Cluster prominence                  &                     -0.064690 \\
		Area density - aligned bounding box &                     -0.039374 \\
		Entropy                             &                      0.033002 \\
		Number of compartments (GMM)        &                     -0.032441 \\
		Asphericity                         &                      0.028517 \\
		Local intensity peak                &                      0.028478 \\
		Global intensity peak               &                     -0.024832 \\
		Intensity range                     &                      0.012509 \\
		Fat.surface                         &                      0.007267 \\
		Major axis length (cm)              &                      0.000000 \\
		Number of voxels of positive value  &                      0.000000 \\
		\bottomrule
	\end{tabular}
\end{table}

Before commenting more in depth the performance of this model it seems appropriate to see at least the other models built with singular features groups, so, when it comes to the clinical features, the results reported in Figure\ref{fig:RocDeathCli} and Table\ref{tab:ParamCli} have been obtained. All the results will be put together for ease in Table \ref{tab:RecapDeath}.

\begin{table}
\caption{Coefficients used in the linear combination estimated by a Lasso regularization relative to the clinical features in modelling \death. All values are in descending order of absolute value\label{tab:ParamCli}}
\centering
	\begin{tabular}{lr}
		\toprule
		Feature Name &   Importance \\
		\midrule
		Intercept          &                      0.178899 \\
		Age (years)        &                      0.116771 \\
		Respiratory Rate   &                      0.082292 \\
		Sex           &                     -0.037591 \\
		Febbre             &                     -0.022923 \\
		Hypertension       &                     -0.000000 \\
		History of smoking &                     -0.000000 \\
		Obesity            &                      0.000000 \\
		\bottomrule
	\end{tabular}
\end{table}

\begin{figure}[htbp]
	\centering
  		\includegraphics[scale=0.25]{ROC_CV_cli.png}
        \caption{ROC curves obtained with crossvalidation procedure using the clinical features alone. In bold is the mean ROC with gray bands of width equal to the standard deviation.\label{fig:RocDeathCli}}
\end{figure}

And finally, considering the radiological features Figure\ref{fig:RocDeathRadiologiche} and Table \ref{tab:ParamRadiologiche} are obtained.

\begin{table}
	\caption{Coefficients used in the linear combination estimated by a Lasso regularization on a linear regression model of death, relative to the radiological features. Values are in descending order of absolute value. \label{tab:ParamRadiologiche}}
		\centering
			\begin{tabular}{lr}
			\toprule
			Feature Name &  Importance \\
			\midrule
			Intercept             &                      0.178899 \\
			Ground-glass          &                     -0.043875 \\
			Lung consolidation    &                      0.038143 \\
			XRayTubeCurrent       &                     -0.017264 \\
			KVP                   &                      0.004995 \\
			Crazy Paving          &                     -0.000000 \\
			Bilateral Involvement &                      0.000000 \\
			SliceThickness        &                      0.000000 \\
			\bottomrule
			\end{tabular}
\end{table}

\begin{figure}[htbp]
	\centering
  		\includegraphics[scale=0.25]{ROC_CV_radiologiche.png}
        \caption{ROC curves obtained with crossvalidation procedure using the radiolocical features alone. As before  in bold is the mean ROC with bands of width equal to the standard deviation.\label{fig:RocDeathRadiologiche}}
\end{figure}

The first thing to notice is that the radiological features, when considered alone, have close to null predictive power. 
This is reasonable for at least part of the features because there is no reason for acquisition parameter to actually influence the outcome of the patient. 
 When it comes to the radiologically determined quantities, such as GGO, Crazy paving, lung consolidation and bilaterlaity even if one would expect these to be relevant their distribution across the dataset is not condusive the good predictions.
 In fact 88\% of patients had GGO, 50\% of all patient had Lung consolidation, 77\% of all patients did not have Crazy paving and 92\% had bilateral involvment.

When it comes to clinical features, category that performs better when considered singularly, nothing ground breaking has been obtained.
Age, Respiratory rate and sex are the most relevant features and are all very much in concordance with what is expected. Finally radiomic features perform slightly worse than the clinical features. To see if the feature obtained are, at least, reasonable a quick explanation is needed. 
This will be done only for the top performing features while deferring to \cite{IBSI} for the complete description:

\begin{itemize}
\item 10$^{th}$ intensity percentile and Intensity based interquartile range are both intensity based statistics.
\item Complexity: A complex image is one that presents many rapid changes in intensity and is heavily non-uniform because it has a lot of primitive components.
\item Cluster Prominence: GLCM feature which measures the symmetry and skewness of the matrix from which it derives. When this is high the image is not symmetric-
\item Area density aligned bounding box: This is a ratio of volume to surface.
\item Entropy: Measures the average quantity of information needed to describe the image. In other words it quantifies randomness in the image, the more random the more info is needed to describe it.
\end{itemize}

To summarize, since all of the features are computed on the whole lung segmentation, it seems that the some information on the distribution of gray levels as well as some textural information inside the whole lung are important.
It also seems that some information on the shape of the organ itself is also relevant.

One would expect that when combining all of the available features, i.e. by building a model using the previous clinical, radiomic and radiological features, the performance should somewhat rise especially given the fact that clinical and radiomic features have almost the same performance. The combined results can be seen in Figure \ref{fig:RocDeathAll} and Table \ref{tab:ParamAll}.

\begin{table}
	\caption{Coefficients used in the linear combination estimated by a Lasso regularization predicting death event relative to all available features features. Values are in descending order of absolute value\label{tab:ParamAll}}
		\centering
			\begin{tabular}{lr}
			\toprule
			Feature Name &  Importance \\
			\midrule
			Intercept                           &                      0.178899 \\
			Age (years)                         &                      0.092963 \\
			Intensity-based interquartile range &                      0.057260 \\
			Respiratory Rate                    &                      0.049603 \\
			Ground-glass                        &                     -0.031423 \\
			Sex\_bin                             &                     -0.028895 \\
			Complexity                          &                     -0.028606 \\
			Lung consolidation                  &                      0.017272 \\
			Febbre                              &                     -0.016933 \\
			XRayTubeCurrent                     &                     -0.016908 \\
			Area density - aligned bounding box &                     -0.009676 \\
			Cluster prominence                  &                     -0.006663 \\
			Fat.surface                         &                      0.004984 \\
			Number of compartments (GMM)        &                     -0.001448 \\
			Local intensity peak                &                      0.000195 \\
			Obesity                             &                      0.000000 \\
			Number of voxels of positive value  &                      0.000000 \\
			Hypertension                        &                      0.000000 \\
			Intensity range                     &                      0.000000 \\
			Global intensity peak               &                     -0.000000 \\
			Asphericity                         &                      0.000000 \\
			Crazy Paving                        &                     -0.000000 \\
			Bilateral Involvement               &                     -0.000000 \\
			SliceThickness                      &                      0.000000 \\
			KVP                                 &                      0.000000 \\
			10th intensity percentile           &                     -0.000000 \\
			Entropy                             &                      0.000000 \\
			History of smoking                  &                     -0.000000 \\
			\bottomrule
\end{tabular}
\end{table}

\begin{figure}[htbp]
	\centering
  		\includegraphics[scale=0.25]{ROC_CV_all.png}
        \caption{ROC curves obtained with crossvalidation procedure using all the available features. Model obtaine with Lasso regularization of a linear regression modelling death\label{fig:RocDeathAll}}
\end{figure}

In spite of what the expectations were, the performance of the combined case seems comparable if not equal to that obtained with clinical variables. To be sure of this claim a Delong test was used to compare pairwise the receiver operator curves and their respective AUCs. 

The null hypothesis of this test is that the two models are the same, hence a p-value smaller than 0.05 means that the curves and their AUCs are statistically different. The results from this analysis can be seen in \ref{fig:delongDeath}

\begin{table}
\caption{Recap table with the performance of the various models predicting on different groups of features predicting \death \label{tab:RecapDeath}}
\centering
\begin{tabular}{l|r}
\toprule
Features used & mean AUC $\pm$std\\
\midrule
Radiomic  & 0.76 $\pm$ 0.09\\
Clinical  &  0.82 $\pm$ 0.11\\
Radiological & 0.61 $\pm$ 0.09\\
All & 0.82 $\pm$ 0.10 \\
\bottomrule
\end{tabular}
\end{table}


\begin{figure}[H]
\centering
  	\subfloat[][Comparison clincal-radiomic]{\includegraphics[width=0.50\linewidth]{Semi_CLIvsRAD_death.png}\label{fig:delongDeathCliRad}}
        \subfloat[][Comparison clinical-all]{\includegraphics[width=0.50\linewidth]{Semi_CLIvsALL_death.png}\label{fig:delongDeathCliAll}}
        \caption{Comparison between ROC curves for clinical vs radiomic curves (a) and clinical vs all (b). The p-values are obtained with a Delong Test }\label{fig:delongDeath}
\end{figure}

As one would have expected the models from radiomic and clinical features are different while the ones built using clinical and all features are not statistically different.
Inspecting the coefficient of the parameters there are a few perplexing things to notice and a few reassuring ones. First of the reassuring facts is that the most relevant clinical features are still relevant in this combined model. 
Then, as one would expect, the radiological features retain some importance when combined with the others. However, when it comes to perplexing behaviours, the most concerning fact is that the radiomic features have mostly lost all relevance in the model which is surely unexpected.

Some possible explanations will be given in the concluding remarks at the end of this  subsection. as well as in the final chapter of the thesis.

\subsection{Classification of patients using Random forests}
For the sake of brevity all of the results will be reported and then discussed. Since RF classifiers use all of the availalble features it is very space consuming to report a table with all of the importances for the radiomic features as well as those used in the models with all the features. These two will be found in the appendix \ref{sec:RFAdditional} while those relative to clinical and radiological features will be reported here in Table \ref{tab:RFimpo}.

\begin{figure}[H]
\centering
  	\subfloat[][Radiomic Features]{\includegraphics[width=0.50\linewidth]{Smote_Death_rad.png}\label{fig:RFDeathRad}}
        \subfloat[][Clinical Features]{\includegraphics[width=0.50\linewidth]{Smote_Death_cli.png}\label{fig:RFDeathCli}}
	\newline
  	\subfloat[][Radiological Features]{\includegraphics[width=0.50\linewidth]{Smote_Death_radiologiche.png}\label{fig:RFDeathRadiological}}
        \subfloat[][All Features]{\includegraphics[width=0.50\linewidth]{Smote_Death_all.png}\label{fig:RFDeathAll}} 
        \caption{Confusion matrices for Random Forest cross-validated predictions after training on Synthetically oversampled data to predict \death. All of the available feature families are reported}\label{fig:RFDeath}
\end{figure}

Even without looking at the ROC curves it's plain to see that the data at hand is proving to be difficult for this model. Much like before radiological features alone are useless.
In evaluating these confusion matrices it should be kept in mind that the data is heavily unbalanced, since only $\sim$15\% of the patient died or were admitted in the ICU.
Even when using SMOTE in the training phase to correct this probelm it seems that the classifiers learns that it's optimal to guess that someone is alive. When it comes to the ROC curves Figure \ref{fig:RFDeathROC} and Table \ref{tab:RecapDeathRF} summarises the results.

\begin{figure}[H]
\centering
  	\subfloat[][Radiomic Features]{\includegraphics[width=0.50\linewidth]{SmoteROC_CV_Death_rad.png}\label{fig:RFDeathRadROC}}
        \subfloat[][Clinical Features]{\includegraphics[width=0.50\linewidth]{SmoteROC_CV_Death_cli.png}\label{fig:RFDeathCliROC}}
	\newline
  	\subfloat[][Radiological Features]{\includegraphics[width=0.50\linewidth]{SmoteROC_CV_Death_radiologiche.png}\label{fig:RFDeathRadiologicalROC}}
        \subfloat[][All Features]{\includegraphics[width=0.50\linewidth]{SmoteROC_CV_Death_all.png}\label{fig:RFDeathAllROC}} 
        \caption{Cross-validated ROC curves built with Random forest classifier predictions of \death. Performances of allvariable families are reported }\label{fig:RFDeathROC}
\end{figure}

\begin{figure}[H]
\centering
  	\subfloat[][Radiological Features]{\begin{tabular}{lr}
				\toprule
				{} &  RF\_importances \\
				\midrule
				Age (years)        &        0.463821 \\
				Respiratory Rate   &        0.287735 \\
				Febbre             &        0.084276 \\
				Sex\_bin            &        0.079571 \\
				Hypertension       &        0.033435 \\
				History of smoking &        0.029404 \\
				Obesity            &        0.021758 \\
				\bottomrule
				\end{tabular}
							}
        \subfloat[][Clinical Features]{
			\begin{tabular}{lr}
				\toprule
				{} &  RF\_importances \\
				\midrule
				XRayTubeCurrent       &        0.800101 \\
				Lung consolidation    &        0.043942 \\
				KVP                   &        0.039768 \\
				Crazy Paving          &        0.032511 \\
				SliceThickness        &        0.031362 \\
				Ground-glass          &        0.030423 \\
				Bilateral Involvement &        0.021893 \\
				HRCT performed        &        0.000000 \\
				\bottomrule
			\end{tabular}
							}
        \caption{Importances estimated by random forest}\label{fig:RFimpo}
\end{figure}


\begin{table}
\caption{Recap table with the performance of the various families of features \label{tab:RecapDeathRF}}
\centering
\begin{tabular}{l|r}
\toprule
Features used & mean AUC $\pm$std\\
\midrule
Radiomic  & 0.65 $\pm$ 0.13\\
Clinical  &  0.64 $\pm$ 0.06\\
Radiological & 0.50 $\pm$ 0.07\\
All & 0.66 $\pm$ 0.8 \\
\bottomrule
\end{tabular}
\end{table}

Once again the curves are evidently not statistically different. This counterintuitive behaviour seems to be constant across the two implemented methods and also across different labels tried. An attempt to explain this phenomenon will be postponed to the next chapter


\section{Predicting and classifying the outcome \icu}
The second step in reporting the results is the comparison of the methods used to predict \icu as clinical outcome.

\subsection{Feature selection through Lasso regularization and clinical outcome prediction using regression}

In trying to predict if the patient will be admitted in the Intensive Care Unit of the hospital the same procedure as before has been used. The ROC curves obtained with the various features are reported in Figure \ref{fig:ICULasso}. Just like before the curve in bold is an average curve obtained by aggregating the ten curves relative to each of the folds used in testing and the gray band represents a $\pm$ 1 standard deviation.
Following the blueprint of the previous subsection, all of the results will be presented and then briefly discussed


\begin{figure}[H]
\centering
  	\subfloat[][Radiomic Features]{\includegraphics[width=0.50\linewidth]{ROC_CV_rad_icu.png}\label{fig:ROCICUrad}}
        \subfloat[][Clinical Features]{\includegraphics[width=0.50\linewidth]{ROC_CV_cli_icu.png}\label{fig:ROCICUcli}}
	\newline
  	\subfloat[][Radiological Features]{\includegraphics[width=0.50\linewidth]{ROC_CV_radiologiche_icu.png}\label{fig:ROCICUradiolocigal}}
        \subfloat[][All Features]{\includegraphics[width=0.50\linewidth]{ROC_CV_all_icu.png}\label{fig:ROCICUall}} 
        \caption{Performances of all the models represented using ROC curves. Each of these has in bold the mean ROC curve over the 10-fold originating from a stratified k-fold cross-validation procedure}\label{fig:ICULasso}
\end{figure}

\begin{table}
\caption{Coefficients used in the linear combination estimated by a Lasso regularization of a model predicting \icu relative to the radiomic features. Values in descending order of modulus \label{tab:ParamRadICU}}
\centering 
	\begin{tabular}{lr}
	\toprule
	Feature Name &   Importance \\
	\midrule
	Intercept                           &                      0.176605 \\
	Number of voxels of positive value  &                      0.160751 \\
	Intensity range                     &                     -0.144834 \\
	Entropy                             &                      0.128999 \\
	Cluster prominence                  &                     -0.122290 \\
	Complexity                          &                     -0.093416 \\
	10th intensity percentile           &                     -0.081133 \\
	Area density - aligned bounding box &                     -0.037373 \\
	Major axis length (cm)              &                     -0.035723 \\
	Dependence count entropy            &                     -0.029603 \\
	Fat.surface                         &                      0.027308 \\
	Asphericity                         &                     -0.023645 \\
	Local intensity peak                &                     -0.019619 \\
	Global intensity peak               &                     -0.016209 \\
	Number of compartments (GMM)        &                     -0.000157 \\
	\bottomrule
	\end{tabular}
\end{table}

\begin{table}
\caption{Coefficients used in the linear combination estimated by a Lasso regularization of a model predicting \icu relative to the clinical features. Values in descending order according to modulus\label{tab:ParamCli}}
\centering
	\begin{tabular}{lr}
	\toprule
	Feature Name &   Importance \\
	\midrule
	Intercept          &                      0.176606 \\
	Respiratory Rate   &                      0.045510 \\
	Febbre             &                      0.038332 \\
	History of smoking &                      0.036888 \\
	Hypertension       &                      0.034547 \\
	Sex\_bin            &                     -0.031504 \\
	Obesity            &                      0.030716 \\
	Age (years)        &                     -0.014646 \\
	\bottomrule
	\end{tabular}
\end{table}


\begin{table}
	\caption{Coefficients used in the linear combination estimated by a Lasso regularization of a model predicting \icu relative to the radiological features\label{tab:ParamRadiologiche}} 
		\centering
			\begin{tabular}{lr}
			\toprule
			Feature Name &   Importance \\
			\midrule
			Intercept             &                  1.766055e-01 \\
			XRayTubeCurrent       &                  0 \\
			Lung consolidation    &                  0\\
			Ground-glass          &                  0\\
			Crazy Paving          &                  0 \\
			Bilateral Involvement &                  0 \\
			SliceThickness        &                 0\\
			KVP                   &                  0 \\
			\bottomrule
			\end{tabular}
\end{table}


\begin{table}
	\caption{Coefficients used in the linear combination estimated by a Lasso regularization of a model predicting \icu relative to all available features features. Values in ascending absolute value order\label{tab:ParamAll}}
		\centering
		\begin{tabular}{lr}
		\toprule
		Feature Name &   Importance \\
		\midrule
		Intercept                           &                      0.176605 \\
		Number of voxels of positive value  &                      0.109947 \\
		Dependence count entropy            &                      0.070527 \\
		Cluster prominence                  &                     -0.069526 \\
		Intensity range                     &                     -0.057924 \\
		Febbre                              &                      0.044149 \\
		Hypertension                        &                      0.039127 \\
		SliceThickness                      &                     -0.037174 \\
		Complexity                          &                     -0.033393 \\
		History of smoking                  &                      0.032812 \\
		Age (years)                         &                     -0.032579 \\
		XRayTubeCurrent                     &                     -0.032182 \\
		Respiratory Rate                    &                      0.028504 \\
		Obesity                             &                      0.027580 \\
		Local intensity peak                &                     -0.026135 \\
		Area density - aligned bounding box &                     -0.023027 \\
		Asphericity                         &                     -0.022999 \\
		Global intensity peak               &                     -0.018280 \\
		Fat.surface                         &                      0.014179 \\
		Sex\_bin                             &                     -0.004316 \\
		Crazy Paving                        &                     -0.003428 \\
		Ground-glass                        &                      0.003077 \\
		Lung consolidation                  &                     -0.001329 \\
		Bilateral Involvement               &                     -0.001047 \\
		Number of compartments (GMM)        &                     -0.000000 \\
		KVP                                 &                      0.000000 \\
		10th intensity percentile           &                     -0.000000 \\
		\bottomrule
		\end{tabular}
\end{table}

Compared to before the performance is definitely worse. 
The radiological features have the same performance of a random variable, which is not that concerning given their expected impact on gravity of the clincial picture of the patient.
Even if superfluous a Delong test was used to confirm that the hypothesis of the curves being equal could not be rejected. When it comes to the relevant features in each model the following can be deduced:

\begin{itemize}
\item For the clinical features all of them have a role in the prediction. The only suprising fact, even if it keeps a certain degree of plausibility, is that age is the less relevant out of the available features when it comes to \icu.
\item None of the radiological features have virtually any impact
\item The radiomic features still value intensity measurements and disorder in the image as primary origins of information. However it seems that shape of the lung now has more relevance in the whole model.
\end{itemize}

\begin{table}
\caption{Recap table with the performance of the various models built for different families of features when predicting \icu \label{tab:RecapICU}}
\centering
\begin{tabular}{l|r}
\toprule
Features used & mean AUC $\pm$std\\
\midrule
Radiomic  & 0.69 $\pm$ 0.13\\
Clinical  &  0.67 $\pm$ 0.14\\
Radiological & 0.49 $\pm$ 0.02\\
All & 0.70 $\pm$ 0.15 \\
\bottomrule
\end{tabular}
\end{table}

\subsection{Classification of patients using Random forests}
Even when using the admission in the ICU the performance of random forests remains pretty much the same when compared to the outcome \death, so the comments would still be the same as before.

\begin{figure}[H]
\centering
  	\subfloat[][Radiomic Features]{\includegraphics[width=0.50\linewidth]{Smote_ICUAdmission_rad.png}\label{fig:RFicuRad}}
        \subfloat[][Clinical Features]{\includegraphics[width=0.50\linewidth]{Smote_ICUAdmission_cli.png}\label{fig:RFicuCli}}
	\newline
  	\subfloat[][Radiological Features]{\includegraphics[width=0.50\linewidth]{Smote_ICUAdmission_radiologiche.png}\label{fig:RFicuRadiological}}
        \subfloat[][All Features]{\includegraphics[width=0.50\linewidth]{Smote_ICUAdmission_all.png}\label{fig:RFicuAll}} 
        \caption{Confusion matrices for Random Forest cross-validated predictions after training on Synthetically oversampled data predicting \icu. All of the available feature families are the reported}\label{fig:RFicu}
\end{figure}


\begin{figure}[H]
\centering
  	\subfloat[][Radiomic Features]{\includegraphics[width=0.50\linewidth]{SmoteROC_CV_ICUAdmission_rad.png}\label{fig:RFicuRadROC}}
        \subfloat[][Clinical Features]{\includegraphics[width=0.50\linewidth]{SmoteROC_CV_ICUAdmission_cli.png}\label{fig:RFicuCliROC}}
	\newline
  	\subfloat[][Radiological Features]{\includegraphics[width=0.50\linewidth]{SmoteROC_CV_ICUAdmission_radiologiche.png}\label{fig:RFicuRadiologicalROC}}
        \subfloat[][All Features]{\includegraphics[width=0.50\linewidth]{SmoteROC_CV_ICUAdmission_all.png}\label{fig:RFicuAllROC}} 
        \caption{Cross-validated ROC curves built with Random forest classifier predictions of \death. Performances of allvariable families are reported}\label{fig:RFicuROC}
\end{figure}


\begin{table}
\caption{Recap table with the performance of the various families of features \label{tab:RecapicuRF}}
\centering
\begin{tabular}{l|r}
\toprule
Features used & mean AUC $\pm$std\\
\midrule
Radiomic  & 0.62 $\pm$ 0.08\\
Clinical  &  0.56 $\pm$ 0.08\\
Radiological & 0.51 $\pm$ 0.11\\
All & 0.64 $\pm$ 0.09\\
\bottomrule
\end{tabular}
\end{table}

\section{Using survival analysis}
Following the preprocessing steps delineated in Materials and Methodologies,  a Cox Proportional-Hazard produces the results presented in Table \ref{tab:CoxResult}.

\begin{table}
\centering
\caption{Results obtained with CoxPH fitter from lifelines library\label{tab:CoxResult}}
\begin{adjustbox}{width=\linewidth}
\begin{tabular}{|lrrrrr|}
\toprule
{} &      coef &  exp(coef) &  se(coef) &           p &  -log2(p) \\
covariate                               &           &            &           &                 &               \\
\midrule
Lung consolidation                     			&  0.166411 &   1.181058 &  0.142506 &        0.242908 &  2.041517 \\
Ground-glass                            			&  0.100946 &   1.106217 &  0.134109 &        0.451619 &  1.146822 \\
Crazy Paving                             			&  0.064744 &   1.066886 &  0.140817 &        0.645680 &  0.631108 \\
Bilateral Involvement                  			& -0.026048 &   0.974288 &  0.121990 &        0.830918 &  0.267222 \\
SliceThickness                          			& -0.008439 &   0.991597 &  0.164518 &        0.959092 &  0.060259 \\
KVP                                        			&  0.376184 &   1.456715 &  0.139983 &        0.007202 &  7.117333 \\
XRayTubeCurrent                     			& -0.272076 &   0.761796 &  0.178915 &        0.128335 &  2.962010 \\
Age (years)                              			& -0.016550 &   0.983587 &  0.160470 &        0.917858 &  0.123657 \\
Hypertension                            			&  0.292450 &   1.339705 &  0.160211 &        0.067940 &  3.879603 \\
History of smoking                    			& -0.083840 &   0.919578 &  0.139121 &         0.546747 &  0.871054 \\
Obesity                                    			&  0.066777 &   1.069057 &  0.170581 &         0.695451 &  0.523979 \\
Respiratory Rate                      			& -0.010716 &   0.989341 &  0.154003 &         0.944525 &  0.082338 \\
Sex\_bin                                 			& -0.366086 &   0.693443 &  0.172651 &         0.033974 &  4.879413 \\
Febbre                                     		  	&  0.115458 &   1.122387 &  0.139878 &         0.409134 &  1.289354 \\
10th intensity percentile             			&  0.324188 &   1.382908 &  0.230611 &        0.159790 &  2.645753 \\
Area density - aligned bounding box     	& -0.169228 &   0.844316 &  0.185602 &        0.361884 &  1.466401 \\
Asphericity                             			& -0.470777 &   0.624517 &  0.174449 &         0.006962 &  7.166236 \\
Cluster prominence                      		& -0.011242 &   0.988821 &  0.234222 &        0.961720 &  0.056312 \\
Complexity                              			&  0.214330 &   1.239032 &  0.248380 &         0.388186 &  1.365179 \\
Global intensity peak                   		&  0.138264 &   1.148279 &  0.165033 &         0.402146 &  1.314210 \\
Intensity range                        			& -0.196751 &   0.821395 &  0.281701 &        0.484901 &  1.044237 \\
Local intensity peak                    			&  0.132819 &   1.142044 &  0.150161 &        0.376419 &  1.409589 \\
Number of compartments (GMM)            	&  0.197884 &   1.218821 &  0.140554 &         0.159164 &  2.651410 \\
Number of voxels of positive value      		&  0.436757 &   1.547680 &  0.284518 &         0.124764 &  3.002721 \\
Fat.surface                             			& -0.425033 &   0.653748 &  0.208060 &        0.041069 &  4.605815 \\
Normalised zone distance non-uniformity 	&  0.589917 &   1.803838 &  0.242437 &         0.014963 &  6.062489 \\
\bottomrule
\end{tabular}
\end{adjustbox}
\end{table}

\begin{figure}
\caption{Graph that represents the coefficient values estimated by the CoxPH model with their respective 95\% confidence intervals}
\includegraphics[width=\textwidth]{Cox_coeffs.png}
\end{figure}

As explained in section \ref{sec:SurvivalAnalysisTheory} the relevant columns are the coeff column, that expressed percentual difference of survival, and the p column, that indicate the significance of the first value.
It turns out that, out of the reduced variables fed to the Cox model, the most relevant are: {\scshape Sex, Asphericity, FatSurface, Normalized zone distance non-uniformity and KVP}.

{\scshape Zone distance non uniformity} measures distribution of zone counts over the different zone distances, it is low when the count relative to the zones are equally distributed along zone distances

{\scshape Asphericity} quantifies how much the segmented region deviates from a sphere.


In this case{\scshape Sex} and {\scshape FatSurface} and {\scshape Normalized zone distance non-uniformity} can be reasonable variables to expect, however {\scshape KVP, Asphericity} seem quite strange.

A score was built automatically using the predict method of the CoxPH fitter and assigned to each patient of the dataset using the previously described cross-validated prediction procedure.
To see if the prediction was representative of differences in the individuated populations first the Kaplan-Meier curves according to thirds in the score distribution were used and then the score was binarized using the 66$^{th}$ percentile in the score distribution as threshold.
The results of this procedures are reported in Figure \ref{fig:KmCoxScore}

\begin{figure}[H]
\centering
  	\subfloat[][Population divided according to score tertiles]{\includegraphics[width=\linewidth]{KM_Cox_groups.png}\label{fig:coxthirds}}
	\newline
        \subfloat[][Population divided in two groups 0-66$^{th}$ percentile and 66$^{th}$ to 100$^{th}$]{\includegraphics[width=\linewidth]{KM_Coxbin.png}\label{fig:coxbin}}
        \caption{Kaplan-Meier curves for pupulations divided using either tertiles in the predicted hazard by the cox model (a) or binarized using 66$^{th}$ percentile as threshold (b) . The prediction on the whole database is obtained with the aforementioned cross-validation procedure }\label{fig:KmCoxScore}
\end{figure}

It can be seen that the groups built in this ways can be used to drive some differences in survival, when binarizing the score obtained with Cox the curves also turn out to be significantly different.

Finally, in order to see if there were differences in treatment or in survival between the two waves of admission, the population was divided in two subgroups according to the date of admission. 
The two groups were representatively called 1$^{st}$ and 2$^{nd}$ wave and the division was drawn on the 20$^{th}$ of July 2020 and the results can be seen in Figure \ref{fig:kmwaves}.

\begin{figure}
\includegraphics[width=\linewidth]{KM_ondate.png}
\caption{Kaplan-Meyer curves for patient admitted before (red curve) and after (blue curve) 20/07/2020 \label{fig:kmwaves}}
\end{figure}

It can be seen that there is a statistical difference in survival between the patients admitted in the first wave vs those admitted in the second. 
Furthermore this difference is quite perplexing as it seems to indicate that people in the second wave died more than people in the first wave, which seems counter-intuitive given that one would expect the experience from the previous wave to improve performance.
The most reasonable explanation for this fact is the change in admission policy as time advanced.
Probably in the first wave, when still little was known on \covid patients, more people were addmitted in less problematic condition whereas in the second wave, having understood better what were the most dangerous cases as well as in an attempt to admitt only those  strictly in need, most of the admitted patients were in more critical condition.

It's also possible that this result that has been obtained could be a symptom of subtle differences in the two \covid manifestations, as if to indicate different variants.
Further analysis in this direction could be a followup work of this thesis.

\chapter{Discussion}
Having presented all of the results obtained in this thesis it has been seen that:

\begin{itemize}
\item The chosen preprocessing method followed by Lasso regularized regressions perform overall well when it comes to predicting either \death or \icu
\item Random Forest classifiers are consistently worse than Lasso regularized regressions, probably mainly due to the unbalancedness of the dataset at hand.
\item Cox proportional Hazard allows us to distinguish at least two groups with statistically different Survival curves.
\item The performance of the models is the same for both clinical and radiomic features, with no statistical difference between the two. Since combining them does not provide any added value, at least in the context of this thesis, the two sets of variables could be considered almost equivalent.
This result is quite perplexing and it could have multiple causes
\end{itemize}

%It is very difficult to quantify what the expectations for each model were a priori, this transposes to difficulties in trying to diagnose problems in the models when considered singularly. However when combining all of the available features it's very strange that no improvement in performance can be achieved.
%This would mean that 8 clinical variables provide the exact same amount of information as the total $\sim$200 features most of which derived from images which, at least ideally, contain much more accurate information. 

%One could surmise that the analysis methods have been implemented in an incorrect way, yet when two methods implemented differently and separately obtain the same result it reinforces the idea that the problem lies somewhere before the analysis.
While it's an interesting result that radiomic features and clinical variables provide the same information, it's quite perplexing that combining does not produce improvements in the performance.

The perplexity seem to arise from the tacit assumption of all the following hypotheses:
%All of these analses started from the following hypotheses:

\begin{enumerate}
\item Clinical labels are informative of the final prognosis of the patient
\item Radiological images contain a lot of useful information
\item Radiomics can extract these information
\item This information is conducive to predicting the prognosis of the patient and is different from that conveyed by clinical variables.
\end{enumerate}

The first three hypotheses are verified with a caveat, the performance of any radiomic pipeline hinges on the quality of the images and of the segmentation procedure performed on them. Since the images used in this thesis were, are and will be used by the hospital in routine processes it's close to impossible that all of them have problems, expecially because of the preliminary screening done before segmentation.

That radiomics can extract useful quantities and that this information can be used for prognosis, specifically in a \covid context, has been discussed in various papers such as \cite{radscore}, \cite{discrimInfluenza}, \cite{Severity}, \cite{MLcovid}, \cite{MLprognostic}, \cite{severityassessment}, \cite{covidscreening} and \cite{accuratediagnosis}.

A first possibility is that when trying to segment lungs affected by \covid the peculiar patterns developed in some way reduce the quality and quantity of information obtainable.
In fact considering that the segmentation method used for this thesis relies on region growth and thresholding methods it's possible that the worst cases end up with unrepresentative segmentations.

Another possibility is that the images are not really representative of the situation of the patient. Since one of the prevailing properties of \covid is the speed with which the clinical picture of the patient can change it's possible that images taken at admission are not as informative of the final prognosis. This problem is, however, unavoidable in the setting of this thesis which has as aim the construction of a model that, exactly at admission, can discriminate between serious and easier cases.

Another possibility is that there are two or more subgroups in the patient cohort and the performance on these is widely different, determiniing an average performance below the expectations.
To diagnose if this is the case a few dimensionality reduction techniques have been used to visualize the data and to prepare for clustering in case of need, all of the results of these procedure will be presented in the Appendix.

To give a qualitative idea of the situation regarding the segmentations $\sim$70 were looked at and evaluated as good, unsure or bad.\footnote{This was done on the first patients in alphabetical order which was considered to be equivalent to random since there is no reasonable motive for surnames to be correlated with segmentation quality.}

Good segmentations are those in which an untrained professional could not see any problems, unsure are those in which there are small inaccuracies, such as lungs that connect in some points and the inclusion of the trachea. 
Finally segmentation were classified as bad in cases with obvious errors, such as part of the intestine being labelled as lung, damages in the lung being labelled as outside tissue or holes in what is supposed to be lung.

\begin{figure}[H]
\centering
  	\subfloat[][Example of segmentation classified as bad]{\includegraphics[width=\linewidth]{Bad_segm.png}\label{fig:badseg}}
	\newline
        \subfloat[][Example of segmentation classified as dubious ]{\includegraphics[width=\linewidth]{Dubiouos_segm.jpg}\label{fig:dubseg}}
        \caption{Example of segmentations being classified as bad (a) and dubious (b). In the first case (a) there is a clear hole in the lung which cannot be exact, in the second (b) there are small portion of outside tissue being labelled as lung as well as the whole trachea.}\label{fig:ExampleSeg}
\end{figure}

The result of this qualitative analysis can be seen in Table \ref{tab:ContingencyTableSegm} in which the incidence of both \death and \icu labels is computed in all possible segmentation categories.

\begin{table}
\centering
\caption{Contingency table with number of \death and \icu labels in all segmentation groups. \label{tab:ContingencyTableSegm}}
\begin{tabular}{llrrr}
\toprule
  & {} & \multicolumn{3}{l}{Subject} \\
  & Segmentation Status &     Bad & Good & Unsure \\
ICU Admission & Death &         &      &        \\
\midrule
0 & 0 &       8 &   11 &     24 \\
  & 1 &       5 &    0 &      6 \\
1 & 0 &       2 &    0 &      2 \\
  & 1 &       1 &    1 &      1 \\
\bottomrule
\end{tabular}
\end{table}

This table suggests that there is large difference of very severe individuals which end up being not perfectly segmented.
It's possible, even if unlikely, that the sample of analysed segmentations is biased in some way. However, given the numbers in Table \ref{tab:ContingencyTableSegm}, it can be reasonable to think that in some of the most important cases, which means those that correspond to 1 labels in \death or \icu, the values of the radiomic features used in this thesis were not the best possible. 

Another thing that can be noted is that in all models that included radiomics feature there are, among the important variables, features that quantify volumetric or shape information regarding the lung.

%A priori it's very difficult to imagine how lung size by itself could determine the prognosis of the patient.
It's possible that some of the most severe cases, which may need a more delicate and ad hoc handling during segmentation, end up being associated to lung shapes that are somewhat different from almost healthy patients.
In this light it's possible that the model is using these shape information, which is in some way altered with respect to reality, to infer the gravity of the situation and is finding something  because the information is somewhat unrealistic in the most severe cases.

%%%%%%%%%%%%%%%%%%%%%%%%%%%%%%%%%%%%% Chapter 4
\chapter{Conclusion}

In this thesis various methods were used in an attempt to predict the prognosis of \covid patients. 

Regularized regression was used to predict clinical outcome using various families of variables to compare the information hidden in each of them and to evaluate if CT exams add any value to a small set of clinical variables.

Random forest classifiers were used with the same aim. As a by-product these two methods can be compared to see which, given the same data, can extract the most information.

Survival analysis was used, mostly by itself, to see if the data could be divided in smaller groups with different survival functions.

In the first two lines of developement it was found that, in the specific case of data available for this thesis, models built on radiomic features perform in a statistically equivalent way to models built from clinical variables.
%This was taken as a (grim) reminder that the quality of the segmentations is very relevant in determining the results obtained by radiomics analysis and that current commercial segmentation techniques based on thresholding and region growing are not yet ready to face the problem posed by \covid pneumonia lungs.

%However the results obtained can be looked at from another perspective.
%The method developed, which also shows potential for improvement, has shown that models using radiomic features have performances on par with those obtained using clinical variables.
This means that, especially in times of system overload due to increased accesses during a pandemic, this system could be integrated in PACS systems of the hospital to bring to attention some of the patients in worst condition.
This, especially thanks to the objectivity of radiomics, the user-independence and, most of all, to the semi-automatic nature of the pipeline could be done on a large scale to aid in hospitals that don't have the facilities , or that lack the personnel, to analyze in detail all cases.
It has also been hypothesised that a more careful handling of images during the segmentation phase, perhaps obtainable with instruments specifically developed for \covid, could lead to imporvements in performances of the model.

Regarding survival analysis it was found that the cohort can be divided in parts using the hazard predicted by a Cox Proportional Hazard model, and it was also found that a time-division driven by the "wave" definition of the pandemic produces signiicanlty different survival curves.

All of that said, some directions in which future works could start from this thesis are:

\begin{itemize}
\item The implementation of a \covid specific segmentation method using the vast amount of available data, similar to what has been done in \cite{Biondi}.
\item It would be interesting to test the pipeline, as well as the deriving models, on manually segmented CT scans to compare the results and performances.
\item Given the statistical difference found in the survival of patients in the first and second waves, defined here as before and after 20/07/2020, it might be very interesting to investigate the causes of this finding and to prospectively continue this analysis with the data from the third wave and eventual next ones that might occur.
\item Using CT scans acquired at different points in the course of the illness it would be very interesting to implement a variation of the pipeline along the direction of delta-radiomics.
\end{itemize}


%%%%%%%%%%%%%%%%%%%%%%%%%%%%%%%%%%%%% Chapter 5
\chapter{Appendix}

\section{Additional Results and complete tables relative to Random Forest}\label{sec:RFAdditional}

Here are the tables with all of the features from random forest. There is no real utility in providing them for all possible feature combination, hence only the importances for all features will be given.
This will be done for both labels used, i.e. Death and ICU Admission. 

\centering
\begin{longtable}{|lr|}
\toprule
Feature Name &  Importance estimated by Random Forest \\
\midrule
Age (years)                                        &        0.056516 \\
CURB65                                             &        0.023775 \\
Intensity histogram quartile coefficient of dis... &        0.021053 \\
Discretised interquartile range                    &        0.017892 \\
Ground-glass                                       &        0.015137 \\
Dependence count entropy                           &        0.014432 \\
Intensity-based interquartile range                &        0.014269 \\
Small zone emphasis                                &        0.014180 \\
Zone size entropy                                  &        0.013493 \\
Normalised zone size non-uniformity                &        0.013321 \\
Skewness                                           &        0.013009 \\
Dependence count energy                            &        0.012736 \\
Information correlation 1                          &        0.011409 \\
Information correlation 2                          &        0.011391 \\
Intensity-based median absolute deviation          &        0.011173 \\
Quartile coefficient of dispersion                 &        0.010367 \\
Respiratory Rate                                   &        0.010176 \\
Entropy                                            &        0.009969 \\
Intensity histogram median absolute deviation      &        0.009406 \\
Run entropy                                        &        0.009264 \\
Volume density - enclosing ellipsoid               &        0.009191 \\
Intensity histogram robust mean absolute deviation &        0.009046 \\
Uniformity                                         &        0.008612 \\
Discretised intensity skewness                     &        0.008386 \\
Intensity-based robust mean absolute deviation     &        0.008134 \\
Maximum histogram gradient intensity               &        0.007806 \\
Grey level variance (GLDZM)                        &        0.007664 \\
Intensity-based mean absolute deviation            &        0.007580 \\
Normalised grey level non-uniformity (NGLDM)       &        0.007455 \\
Fat.surface                                        &        0.007220 \\
Febbre                                             &        0.007157 \\
Sum entropy                                        &        0.006892 \\
Local intensity peak                               &        0.006887 \\
Minor axis length (cm)                             &        0.006875 \\
Area density - enclosing ellipsoid                 &        0.006777 \\
Grey level variance (GLSZM)                        &        0.006726 \\
Angular second moment                              &        0.006453 \\
Cluster shade                                      &        0.006377 \\
XRayTubeCurrent                                    &        0.006247 \\
Max value                                          &        0.006077 \\
Zone distance non-uniformity                       &        0.005951 \\
Normalised zone distance non-uniformity            &        0.005922 \\
Small distance emphasis                            &        0.005790 \\
Cluster prominence                                 &        0.005772 \\
RECIST (cm)                                        &        0.005728 \\
Large distance high grey level emphasis            &        0.005609 \\
Normalised grey level non-uniformity (GLRLM)       &        0.005479 \\
Low dependence emphasis                            &        0.005422 \\
Small distance low grey level emphasis             &        0.005405 \\
Normalised grey level non-uniformity (GLSZM)       &        0.005318 \\
Grey level non-uniformity (NGLDM)                  &        0.005307 \\
Volume density - convex hull                       &        0.005301 \\
Volume at intensity fraction 90\%                   &        0.005267 \\
Large distance emphasis                            &        0.005247 \\
Normalised homogeneity                             &        0.005180 \\
Dependence count non-uniformity                    &        0.005174 \\
Small zone low grey level emphasis                 &        0.005110 \\
Number of grey levels                              &        0.005006 \\
Area density - convex hull                         &        0.004984 \\
Low grey level zone emphasis.1                     &        0.004983 \\
10th intensity percentile                          &        0.004967 \\
Intensity histogram mean absolute deviation        &        0.004965 \\
Intensity median value                             &        0.004960 \\
Discretised intensity kurtosis                     &        0.004954 \\
Energy                                             &        0.004950 \\
High dependence low grey level emphasis            &        0.004895 \\
Integrated intensity                               &        0.004888 \\
Small distance high grey level emphasis            &        0.004863 \\
Normalised inverse difference                      &        0.004814 \\
Zone distance entropy                              &        0.004801 \\
Normalised grey level non-uniformity (GLDZM)       &        0.004749 \\
Difference average                                 &        0.004743 \\
Thresholded area intensity peak (50\%)              &        0.004735 \\
Centre of mass shift (cm)                          &        0.004683 \\
Minimum histogram gradient                         &        0.004634 \\
Number of voxels                                   &        0.004592 \\
Low grey level zone emphasis                       &        0.004470 \\
Area density - oriented bounding box               &        0.004465 \\
Volume density - aligned bounding box              &        0.004453 \\
High dependence emphasis                           &        0.004453 \\
Intensity-based coefficient of variation           &        0.004442 \\
Thresholded area intensity peak (75\%)              &        0.004426 \\
Discretised intensity uniformity                   &        0.004342 \\
Low grey level count emphasis                      &        0.004310 \\
Grey level non-uniformity (GLDZM)                  &        0.004261 \\
Contrast (GLCM)                                    &        0.004247 \\
Difference entropy                                 &        0.004174 \\
Kurtosis                                           &        0.004151 \\
Grey level non-uniformity (GLRLM)                  &        0.004114 \\
Number of compartments (GMM)                       &        0.004110 \\
Intensity-based energy                             &        0.004093 \\
Small zone high grey level emphasis                &        0.004086 \\
Least axis length (cm)                             &        0.004086 \\
Intensity histogram mode                           &        0.004073 \\
Volume density - oriented bounding box             &        0.004064 \\
Inverse variance                                   &        0.004055 \\
Difference variance                                &        0.004024 \\
Surface to volume ratio                            &        0.003966 \\
Run length variance                                &        0.003913 \\
Variance                                           &        0.003910 \\
Correlation                                        &        0.003908 \\
Muscle.surface                                     &        0.003907 \\
High grey level zone emphasis                      &        0.003879 \\
Number of voxels of positive value                 &        0.003857 \\
Inverse elongation                                 &        0.003853 \\
Cluster tendency                                   &        0.003820 \\
Intensity range                                    &        0.003804 \\
Normalised run length non-uniformity               &        0.003799 \\
Large zone high grey level emphasis                &        0.003776 \\
Long run low grey level emphasis                   &        0.003766 \\
Area density - aligned bounding box                &        0.003687 \\
Zone percentage (GLDZM)                            &        0.003660 \\
Asphericity                                        &        0.003657 \\
Grey level variance (NGLDM)                        &        0.003643 \\
Intensity at volume fraction 90\%                   &        0.003617 \\
Volume at intensity fraction 10\%                   &        0.003610 \\
Major axis length (cm)                             &        0.003604 \\
Low dependence low grey level emphasis             &        0.003570 \\
Run length non-uniformity                          &        0.003548 \\
Strength                                           &        0.003459 \\
Long run high grey level emphasis                  &        0.003433 \\
Mean discretised intensity                         &        0.003413 \\
Low dependence high grey level emphasis            &        0.003409 \\
Dissimilarity                                      &        0.003395 \\
High grey level count emphasis                     &        0.003394 \\
SliceThickness                                     &        0.003389 \\
Grey level non-uniformity (GLSZM)                  &        0.003381 \\
Volume fraction difference between intensity fr... &        0.003381 \\
Grey level variance (GLRLM)                        &        0.003369 \\
Short run low grey level emphasis                  &        0.003347 \\
Maximum histogram gradient                         &        0.003338 \\
High dependence high grey level emphasis           &        0.003321 \\
Compactness 2                                      &        0.003317 \\
Long run emphasis                                  &        0.003316 \\
Autocorrelation                                    &        0.003261 \\
Joint maximum                                      &        0.003254 \\
Global intensity peak                              &        0.003237 \\
Sum average                                        &        0.003233 \\
Low grey level run emphasis                        &        0.003187 \\
Dependence count variance                          &        0.003182 \\
Intensity at volume fraction 10\%                   &        0.003128 \\
Large distance low grey level emphasis             &        0.003125 \\
Zone percentage (GLSZM)                            &        0.003088 \\
Intensity fraction difference between volume fr... &        0.003034 \\
Zone distance variance                             &        0.002949 \\
Maximum 3D diameter (cm)                           &        0.002940 \\
Normalized dependence count non-uniformity         &        0.002937 \\
Inverse difference                                 &        0.002912 \\
Intensity histogram coefficient of variation       &        0.002872 \\
Coarseness                                         &        0.002836 \\
Run percentage                                     &        0.002805 \\
Flatness                                           &        0.002788 \\
Standard deviation                                 &        0.002760 \\
Joint variance                                     &        0.002714 \\
Busyness                                           &        0.002711 \\
Intensity mean value                               &        0.002706 \\
Homogeneity                                        &        0.002698 \\
Large zone low grey level emphasis                 &        0.002665 \\
Joint average                                      &        0.002614 \\
KVP                                                &        0.002597 \\
90th discretised intensity percentile              &        0.002525 \\
90th intensity percentile                          &        0.002523 \\
Contrast (NGTDM)                                   &        0.002511 \\
Joint Entropy                                      &        0.002488 \\
Spherical disproportion                            &        0.002455 \\
Sphericity                                         &        0.002423 \\
Discretised intensity standard deviation           &        0.002377 \\
Compactness 1                                      &        0.002372 \\
Crazy Paving                                       &        0.002324 \\
Area under the IVH curve                           &        0.002259 \\
High grey level run emphasis                       &        0.002258 \\
Complexity                                         &        0.002223 \\
Short run emphasis                                 &        0.002206 \\
Discretised intensity variance                     &        0.002196 \\
Large zone emphasis                                &        0.002158 \\
Short run high grey level emphasis                 &        0.002158 \\
High grey level zone emphasis.1                    &        0.002006 \\
Median discretised intensity                       &        0.001935 \\
Min value                                          &        0.001904 \\
Quadratic mean                                     &        0.001870 \\
Obesity                                            &        0.001766 \\
Discretised intensity entropy                      &        0.001680 \\
Sex\_bin                                            &        0.001662 \\
Minimum histogram gradient intensity               &        0.001547 \\
Sum variance                                       &        0.001496 \\
Lung consolidation                                 &        0.000751 \\
Bilateral Involvement                              &        0.000694 \\
History of smoking                                 &        0.000513 \\
Hypertension                                       &        0.000493 \\
Discretised max value                              &        0.000000 \\
Discretised min value                              &        0.000000 \\
Discretized intensity range                        &        0.000000 \\
Dependence count percentage                        &        0.000000 \\
Number of grey levels after quantization           &        0.000000 \\
HRCT performed                                     &        0.000000 \\
\bottomrule
\caption{Importances determined by RandomForest predicting death using all available features. The values are in descending order.} % needs to go inside longtable environment
\label{tab:RFimpoAllDeath}
\end{longtable}

\centering
\begin{longtable}{|lr|}
\toprule
{} &  RF\_importances \\
\midrule
Age (years)                                        &        0.043174 \\
Sex\_bin                                            &        0.020316 \\
Fat.surface                                        &        0.017942 \\
Dependence count entropy                           &        0.017538 \\
Dependence count energy                            &        0.015580 \\
Respiratory Rate                                   &        0.012740 \\
Intensity histogram quartile coefficient of dis... &        0.012675 \\
Flatness                                           &        0.012201 \\
Discretised interquartile range                    &        0.012029 \\
Small zone high grey level emphasis                &        0.011807 \\
Run entropy                                        &        0.010264 \\
Intensity histogram median absolute deviation      &        0.010097 \\
Least axis length (cm)                             &        0.009921 \\
Muscle.surface                                     &        0.009778 \\
Dependence count variance                          &        0.009557 \\
Angular second moment                              &        0.009544 \\
Quartile coefficient of dispersion                 &        0.009437 \\
Large distance high grey level emphasis            &        0.009423 \\
Joint Entropy                                      &        0.009381 \\
Low dependence high grey level emphasis            &        0.009338 \\
SliceThickness                                     &        0.009026 \\
Inverse elongation                                 &        0.008748 \\
Intensity-based interquartile range                &        0.008110 \\
Information correlation 2                          &        0.008001 \\
Run length variance                                &        0.007324 \\
Dependence count non-uniformity                    &        0.007150 \\
Energy                                             &        0.006935 \\
Global intensity peak                              &        0.006818 \\
Normalized dependence count non-uniformity         &        0.006794 \\
RECIST (cm)                                        &        0.006689 \\
Maximum 3D diameter (cm)                           &        0.006610 \\
Lung consolidation                                 &        0.006506 \\
Centre of mass shift (cm)                          &        0.006447 \\
Max value                                          &        0.006395 \\
Information correlation 1                          &        0.006361 \\
Compactness 1                                      &        0.006337 \\
Run percentage                                     &        0.006314 \\
Long run emphasis                                  &        0.006191 \\
Short run emphasis                                 &        0.006061 \\
Zone distance non-uniformity                       &        0.006042 \\
Sphericity                                         &        0.005936 \\
Normalised run length non-uniformity               &        0.005929 \\
Autocorrelation                                    &        0.005853 \\
High grey level zone emphasis.1                    &        0.005797 \\
Volume density - convex hull                       &        0.005778 \\
Integrated intensity                               &        0.005719 \\
Volume density - oriented bounding box             &        0.005678 \\
Intensity histogram robust mean absolute deviation &        0.005671 \\
Volume at intensity fraction 90\%                   &        0.005601 \\
Normalised homogeneity                             &        0.005591 \\
Inverse difference                                 &        0.005562 \\
Local intensity peak                               &        0.005533 \\
Area density - oriented bounding box               &        0.005528 \\
Inverse variance                                   &        0.005505 \\
Intensity-based energy                             &        0.005463 \\
Crazy Paving                                       &        0.005460 \\
Sum entropy                                        &        0.005449 \\
Homogeneity                                        &        0.005447 \\
Small distance low grey level emphasis             &        0.005413 \\
Asphericity                                        &        0.005396 \\
Thresholded area intensity peak (50\%)              &        0.005375 \\
Minimum histogram gradient                         &        0.005369 \\
High dependence high grey level emphasis           &        0.005369 \\
Contrast (GLCM)                                    &        0.005365 \\
Zone distance variance                             &        0.005334 \\
Surface to volume ratio                            &        0.005209 \\
Volume density - aligned bounding box              &        0.005162 \\
Spherical disproportion                            &        0.005159 \\
Sum average                                        &        0.005132 \\
High dependence low grey level emphasis            &        0.005112 \\
Area density - convex hull                         &        0.005111 \\
Grey level variance (GLDZM)                        &        0.005107 \\
Compactness 2                                      &        0.004996 \\
Number of voxels of positive value                 &        0.004984 \\
Discretised intensity uniformity                   &        0.004979 \\
KVP                                                &        0.004974 \\
Cluster shade                                      &        0.004964 \\
Thresholded area intensity peak (75\%)              &        0.004958 \\
Grey level non-uniformity (GLDZM)                  &        0.004918 \\
Normalised zone distance non-uniformity            &        0.004912 \\
Number of grey levels                              &        0.004905 \\
Grey level non-uniformity (NGLDM)                  &        0.004786 \\
Dissimilarity                                      &        0.004767 \\
Large zone high grey level emphasis                &        0.004756 \\
Complexity                                         &        0.004710 \\
Cluster prominence                                 &        0.004679 \\
Low dependence low grey level emphasis             &        0.004673 \\
Area density - aligned bounding box                &        0.004666 \\
Long run high grey level emphasis                  &        0.004659 \\
Grey level variance (GLSZM)                        &        0.004627 \\
Normalised zone size non-uniformity                &        0.004610 \\
Strength                                           &        0.004605 \\
Normalised grey level non-uniformity (NGLDM)       &        0.004563 \\
Difference variance                                &        0.004546 \\
Correlation                                        &        0.004544 \\
CURB65                                             &        0.004524 \\
Normalised grey level non-uniformity (GLRLM)       &        0.004522 \\
Small zone emphasis                                &        0.004512 \\
Volume at intensity fraction 10\%                   &        0.004447 \\
Large distance emphasis                            &        0.004423 \\
Minor axis length (cm)                             &        0.004406 \\
Zone distance entropy                              &        0.004374 \\
XRayTubeCurrent                                    &        0.004373 \\
Area density - enclosing ellipsoid                 &        0.004333 \\
Small distance high grey level emphasis            &        0.004326 \\
Contrast (NGTDM)                                   &        0.004271 \\
Low dependence emphasis                            &        0.004255 \\
Short run high grey level emphasis                 &        0.004209 \\
Small zone low grey level emphasis                 &        0.004182 \\
Difference average                                 &        0.004180 \\
Intensity range                                    &        0.004178 \\
High grey level zone emphasis                      &        0.004157 \\
Intensity-based robust mean absolute deviation     &        0.004130 \\
Intensity histogram coefficient of variation       &        0.004124 \\
Difference entropy                                 &        0.004122 \\
Major axis length (cm)                             &        0.004113 \\
Volume fraction difference between intensity fr... &        0.004113 \\
Low grey level count emphasis                      &        0.004092 \\
Intensity median value                             &        0.004051 \\
Uniformity                                         &        0.004049 \\
Grey level non-uniformity (GLRLM)                  &        0.004033 \\
High grey level run emphasis                       &        0.004027 \\
Number of voxels                                   &        0.004015 \\
Joint variance                                     &        0.003956 \\
Run length non-uniformity                          &        0.003941 \\
Discretised intensity entropy                      &        0.003905 \\
Zone percentage (GLDZM)                            &        0.003893 \\
Large zone low grey level emphasis                 &        0.003886 \\
Sum variance                                       &        0.003878 \\
Low grey level zone emphasis                       &        0.003853 \\
Zone percentage (GLSZM)                            &        0.003821 \\
High grey level count emphasis                     &        0.003811 \\
Busyness                                           &        0.003795 \\
High dependence emphasis                           &        0.003757 \\
Grey level non-uniformity (GLSZM)                  &        0.003727 \\
Large distance low grey level emphasis             &        0.003638 \\
Volume density - enclosing ellipsoid               &        0.003633 \\
Zone size entropy                                  &        0.003558 \\
Joint maximum                                      &        0.003558 \\
Discretised intensity skewness                     &        0.003552 \\
Skewness                                           &        0.003496 \\
Grey level variance (NGLDM)                        &        0.003480 \\
Area under the IVH curve                           &        0.003470 \\
Normalised grey level non-uniformity (GLDZM)       &        0.003468 \\
Intensity histogram mean absolute deviation        &        0.003431 \\
Normalised inverse difference                      &        0.003320 \\
Short run low grey level emphasis                  &        0.003299 \\
Mean discretised intensity                         &        0.003293 \\
Low grey level zone emphasis.1                     &        0.003285 \\
Discretised intensity kurtosis                     &        0.003275 \\
Small distance emphasis                            &        0.003148 \\
Joint average                                      &        0.003141 \\
Intensity-based median absolute deviation          &        0.003112 \\
90th discretised intensity percentile              &        0.003094 \\
Large zone emphasis                                &        0.003052 \\
90th intensity percentile                          &        0.003030 \\
10th intensity percentile                          &        0.003021 \\
Low grey level run emphasis                        &        0.002930 \\
Kurtosis                                           &        0.002860 \\
Cluster tendency                                   &        0.002743 \\
Intensity at volume fraction 10\%                   &        0.002725 \\
Grey level variance (GLRLM)                        &        0.002673 \\
Long run low grey level emphasis                   &        0.002640 \\
Intensity-based coefficient of variation           &        0.002634 \\
Min value                                          &        0.002620 \\
Intensity mean value                               &        0.002595 \\
Entropy                                            &        0.002562 \\
Normalised grey level non-uniformity (GLSZM)       &        0.002561 \\
Variance                                           &        0.002516 \\
Minimum histogram gradient intensity               &        0.002509 \\
Discretised intensity standard deviation           &        0.002474 \\
Maximum histogram gradient intensity               &        0.002467 \\
Standard deviation                                 &        0.002411 \\
Maximum histogram gradient                         &        0.002410 \\
Intensity at volume fraction 90\%                   &        0.002390 \\
Quadratic mean                                     &        0.002362 \\
Intensity fraction difference between volume fr... &        0.002215 \\
Intensity-based mean absolute deviation            &        0.002203 \\
Discretised intensity variance                     &        0.001964 \\
Intensity histogram mode                           &        0.001620 \\
Coarseness                                         &        0.001438 \\
Median discretised intensity                       &        0.001337 \\
Number of compartments (GMM)                       &        0.001055 \\
Obesity                                            &        0.000973 \\
History of smoking                                 &        0.000896 \\
Bilateral Involvement                              &        0.000893 \\
Ground-glass                                       &        0.000859 \\
Hypertension                                       &        0.000497 \\
Febbre                                             &        0.000358 \\
HRCT performed                                     &        0.000000 \\
Discretized intensity range                        &        0.000000 \\
Discretised min value                              &        0.000000 \\
Discretised max value                              &        0.000000 \\
Dependence count percentage                        &        0.000000 \\
Number of grey levels after quantization           &        0.000000 \\
\bottomrule
\caption{Importances determined by RandomForest predicting ICU Admission using all available features. The values are in descending order.} % needs to go inside longtable environment
\label{tab:RFimpoRad}
\end{longtable}

\section{Using Dimensionality reduction to further investigate the dataset}
For these analyses the data was always fed into a standard scaler before applying the technique of choice, furthermore a custom gravity score by classifying as 4 the dead individuals and then by assigning a progressive score form 1 to 3 by looking at the time of permanence was built as follows:

\begin{enumerate}
\item Gravity 1: Survived individuals with permanence from 0$^{th}$ percentile to 25$^{th}$ percentile
\item Gravity 2:Survived individuals with permanence from 25$^{th}$ percentile to 75$^{th}$ percentile
\item Gravity 3:Survived individuals with permanence from 75$^{th}$ percentile to 100$^{th}$ percentile
\item Gravity 4: Dead individuals without regard for permanence in the hospital
\end{enumerate}

\begin{figure}[htbp]
	\centering 
  		\includegraphics[width = \linewidth]{Scale_umap_gravityquantilies2575.png}
        \caption{Possible combination for umap hyperparameters "number of neighbours" and "minimum distance". Color coding is done with aforementioned gravity score and all the features, i.e. clinical radiomic and radiological, were used.  \label{fig:hyperparam_umap}}
\end{figure}


Also, before proceding, the hyperparameter space for umap was explored since it's the method that allows the most control over rather intuitive parameters.
Changing the value of the minimum distance of points in the final space from 0 to 0.99 changes how the structure is projected, while changing the number of neighbours changes how much the local or global structure of the data influences the final projection.
 Some of the combinations of these parameters can be seen in Figure \ref{fig:hyperparam_umap}


\subsection{Explaining total variance using PCA}

Starting from PCA, the data was reduced to either two or three dimensions considering clinical and radiomic features, both separated and together.
In this first example there seems to be a kind of left leaning polarization of the dead individuals, however there are no clear separations in the data.

\begin{figure}[h!]
\centering
  		\includegraphics[width=\linewidth]{PCA2_Death.png}
        \caption{2-Principal Component on clinical features.  \label{fig:PCA2_death}}
\end{figure}

 Working on the clinical dataset it can also be noted that the first two components of the PCA explain only 36$\%$ of the total variance. This leads to the conclusion that changes in the data cannot be explained by a single, nor a few, features or linear combination thereof. The next approach was  using the first three principal components using various labels available, most relevant of which being ICU admission, Death and Gravity score.

\begin{figure}[h!]
  		\includegraphics[width=\textwidth]{PCA3_all_Death.png}\label{PCA3_all_death}
  		\caption{3D PCA of whole dataset, colored with death label}
\end{figure}

\begin{figure}[h!]
  		\includegraphics[width=\textwidth]{PCA3_all_ICU.png}\label{PCA3_all_ICU}
  		\caption{3D PCA of whole dataset, colored with ICU Admission label}
\end{figure}

In all cases it seems like introducing the radiomic features causes the loss of the polarization structure that could be seen in the PCA on the clinical dataset alone in fig\ref{fig:PCA2_death}.
 
\begin{figure}[h!]
  		\includegraphics[width=\textwidth]{PCA3_all_Gravity.png}\label{PCA3_all_gravity}
          \caption{Comparison between various colour labels of the top 3 principal components for the entire dataset. Note that to explain 95$\%$ of the variance 27 components would be needed}
\end{figure}

Since there are no visible clusters proceeding with cluster analysis would mean incurring in the risk of finding non meaningful results so it seemed appropriate to try other dimensionality reduction techniques.

\subsection{Exploring data structure with UMAP}
The next technique tried was unsupervised Umap. Following the conclusions derived from fig:\ref{fig:hyperparam_umap} the number of neighbours was set to 10 and the minimum distance was set to 0.
Once again the comparison were made between clinical and radiomic dataset as well as different possible labellings. 
Starting from the clinical dataset, without reporting all labels used, it's clear to see that the dataset seems to indicate very local well separated structures which don't seem correlated to gravity outcome

\begin{figure}[htbp]
  		\includegraphics[width=\textwidth]{umap3_cli_Gravity.png}\label{umap3_cli_gravity}
          \caption{3D umap of clinical dataset, colored based on gravity}
\end{figure}

There are 9 well defined groups which don't seem to be correlated to any of the available labels. 
The dimension of these group is also very prohibitive if thinking of further analyses since groups of 35-50 people in a dataset with 15$\%$ mortality rate would mostly be very unbalanced if they were to be used for classification.
However if the introduction of radiomic features were to unite some of these groups then this embedding could be meaningfully used for analysis. 
Looking at the 3D embedding for the whole dataset, the results are:

\begin{figure}[htbp]
  		\includegraphics[width=\textwidth]{umap3_all_death.png}\label{umap3_all_death}
          \caption{3D umap of whole dataset, colored based on death}
\end{figure}

\begin{figure}[htbp]
  		\includegraphics[width=\textwidth]{umap3_all_ICU.png}\label{umap3_all_ICU}
          \caption{3D umap of whole dataset, colored based on ICU admission}
\end{figure}

\begin{figure}[htbp]
  		\includegraphics[width=\textwidth]{umap3_all_Gravity.png}\label{umap3_all_Gravity}
          \caption{3D umap of whole dataset, colored based on gravity}
\end{figure}

Once again the introduction of the radiomic feature seems to be a confounding factor in the seemingly clear-cut order present in the clinical dataset alone.
There seems to be a well connected structure, which makes sense because umap sets out with the objective of preserving said structure.
However since the variables of interest as label are Death, ICU admission or some kind of combination of them with hospital permanence there seems to be no visual correlation between structure and label.
As such the next dimensionality reduction was tried to see if it yielded better results.
 
\subsection{Predicting clinical outcome using PLS-DA}
Moving on from unsupervised methods to a supervised one, PLS-DA was used giving as label both death and ICU using both whole dataset, and singularly radiomic or clinical features. 
Starting from the clinical features alone, predicting on death Figure~\ref{PLSDA-Clinical} can be obtained.

\begin{figure*}[htbp]
  		\includegraphics[width=1\textwidth]{Clinical_dataset.png}
          \caption{PLS-DA predicting on death coloured with age and hypertension\label{PLSDA-Clinical}}
\end{figure*}

In Figure~\ref{PLSDA-Clinical} there are a few things to note. 
The first is the presence, in the top plot, of an outlier which, since PLS-DA is based on minimization of least squares, can ruin a lot the performance of the procedure. 
For this reason in the second plot the outlier was removed and the algorithm was run again on the cleaned data.
The second thing to notice is the coloring used which, in the first plot, was used to highlight that along the one of the two latent variables the data is roughly distributed depending on age while, in the second plot, was used to highlight that the algorithm is able to perfectly separate the subjects with hypertension from those without it.
However, by looking at the same embedding labelled with death and ICU admission fig \ref{PLSDA-Clinical-death} can be obtained

\begin{figure}[htbp]
  		\includegraphics[width=\textwidth]{PLSDA_cli_death.png}
          \caption{PLS-DA predicting on death coloured with death(bottom) and ICU Admission(top)\label{PLSDA-Clinical-death}}
\end{figure}

It's clear to see that, when predicting on death, the PLS-DA algorithm doesn't find any behaviour relevant for ICU admission.
It's also clear that there is at least a pattern of points labeled as dead being towards the right of the image, this can be easily explained by looking at how the ages are distributed in the first plot of fig: \ref{PLSDA-Clinical}. 
From this it's possible to deduce that older individuals tend to die more and that hypertension does not seem to be relevant when considering death as a clinical outcome.
If necessary the PLS-DA algorithm allows also to see the weights given to the features in predicting the label. At least for the clinical dataset, which has a reasonable number of features, it's interesting to report it ordering the coefficients by descending absolute value: 

\begin{table}
\caption{PLS-DA feature weights in prediction on death using clinical features}
\centering
	\begin{tabular}{|l|r|}
	\hline
	Feature Name &         Importance \\
	\hline
	 Respiratory Rate &  0.120206 \\
	 Age (years) &  0.116305 \\
	 Obesity &  0.004293 \\
	 Hypertension & -0.004626 \\
	 History of smoking & -0.012314 \\
	 Febbre & -0.045431 \\
	 Sex\_bin & -0.054947 \\
	\hline
	\end{tabular}
\end{table}

Doing the exact same procedure on the whole dataset, which means by including the radiomic features, Figure~\ref{fig:whole_dataset} can be obtained.

\begin{figure}[htbp]
  		\includegraphics[width=\textwidth]{Whole_dataset.png}
          \caption{PLS-DA predicting on death coloured with death(top) and hypertension(bottom) on whole dataset\label{fig:whole_dataset}}
\end{figure}



\begin{figure*}[htbp]
  		\includegraphics[width=1\textwidth]{PLSDA_cli_ICU_LV.png}
          \caption{PLS-DA predicting on death coloured with Age and respiratory rate on clinical features\label{fig:PLSDA-ICU-LV}}
\end{figure*}

Once again adding the radiomic features has evidently introduced noise in the system, which no longer displays any kind of behaviour, pattern nor separation.
Doing the same analysis but using ICU Admission as a label Figure~\ref{fig:PLSDA-ICU-LV} can be obtained. 

Now the colors have been chosen to highlight that respiratory rate and age have the main role in determining the latent variables. However, in this case, there doesn't appear to be a clear cut distinction as it happened before with hypertension.
Looking at how the points scatter by coloring them according to the two interesting clinical labels the following figure can be obtained:

\begin{figure}[H]
  		\includegraphics[width=\textwidth]{PLSDA_cli_ICU.png}
          \caption{PLS-DA predicting on death(bottom) coloured with death and ICU admission(top) on clinical features\label{fig:PLSDA-ICU}}
\end{figure}

Finally, introducing the radiomic features in the analysis the usual effect of reducing separation can be seen can be seen in the figure below:

\begin{figure}[H]
  		\includegraphics[width=\textwidth]{PLSDA_all_ICU.png}
          \caption{PLS-DA predicting on death coloured with death(bottom) and ICU admission(top) on all available features\label{fig:PLSDA-ICU-all}}
\end{figure}
 

%-------------------------------------REFERENCES

\nocite{*}
\bibliographystyle{abbrv}
\bibliography{biblio}

\end{document}
